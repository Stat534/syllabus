\documentclass[11pt,]{article}
\usepackage[margin=1in]{geometry}
\newcommand*{\authorfont}{\fontfamily{phv}\selectfont}
\usepackage[]{mathpazo}
\usepackage{abstract}
\renewcommand{\abstractname}{}    % clear the title
\renewcommand{\absnamepos}{empty} % originally center
\newcommand{\blankline}{\quad\pagebreak[2]}

\providecommand{\tightlist}{%
  \setlength{\itemsep}{0pt}\setlength{\parskip}{0pt}} 
\usepackage{longtable,booktabs}

\usepackage{parskip}
\usepackage{titlesec}
\titlespacing\section{0pt}{12pt plus 4pt minus 2pt}{6pt plus 2pt minus 2pt}
\titlespacing\subsection{0pt}{12pt plus 4pt minus 2pt}{6pt plus 2pt minus 2pt}

\titleformat*{\subsubsection}{\normalsize\itshape}

\usepackage{titling}
\setlength{\droptitle}{-.25cm}

%\setlength{\parindent}{0pt}
%\setlength{\parskip}{6pt plus 2pt minus 1pt}
%\setlength{\emergencystretch}{3em}  % prevent overfull lines 

\usepackage[T1]{fontenc}
\usepackage[utf8]{inputenc}

\usepackage{fancyhdr}
\pagestyle{fancy}
\usepackage{lastpage}
\renewcommand{\headrulewidth}{0.3pt}
\renewcommand{\footrulewidth}{0.0pt} 
\lhead{}
\chead{}
\rhead{\footnotesize STAT 534: Spatial Statistics -- Spring 2021}
\lfoot{}
\cfoot{\small \thepage/\pageref*{LastPage}}
\rfoot{}

\fancypagestyle{firststyle}
{
\renewcommand{\headrulewidth}{0pt}%
   \fancyhf{}
   \fancyfoot[C]{\small \thepage/\pageref*{LastPage}}
}

%\def\labelitemi{--}
%\usepackage{enumitem}
%\setitemize[0]{leftmargin=25pt}
%\setenumerate[0]{leftmargin=25pt}




\makeatletter
\@ifpackageloaded{hyperref}{}{%
\ifxetex
  \usepackage[setpagesize=false, % page size defined by xetex
              unicode=false, % unicode breaks when used with xetex
              xetex]{hyperref}
\else
  \usepackage[unicode=true]{hyperref}
\fi
}
\@ifpackageloaded{color}{
    \PassOptionsToPackage{usenames,dvipsnames}{color}
}{%
    \usepackage[usenames,dvipsnames]{color}
}
\makeatother
\hypersetup{breaklinks=true,
            bookmarks=true,
            pdfauthor={ ()},
             pdfkeywords = {},  
            pdftitle={STAT 534: Spatial Statistics},
            colorlinks=true,
            citecolor=blue,
            urlcolor=blue,
            linkcolor=magenta,
            pdfborder={0 0 0}}
\urlstyle{same}  % don't use monospace font for urls


\setcounter{secnumdepth}{0}





\usepackage{setspace}

\title{STAT 534: Spatial Statistics}
\author{Andrew Hoegh}
\date{Spring 2021}


\begin{document}  

		\maketitle
		
	
		\thispagestyle{firststyle}

%	\thispagestyle{empty}


	\noindent \begin{tabular*}{\textwidth}{ @{\extracolsep{\fill}} lr @{\extracolsep{\fill}}}


E-mail: \texttt{\href{mailto:andrew.hoegh@montana.edu}{\nolinkurl{andrew.hoegh@montana.edu}}} & Web: \href{http://stat534.github.io}{\tt stat534.github.io}\\
Office Hours: TBD  &  Class Hours: MWF 12:00-12:50\\
Office: Wilson Hall 2-241  & Class Room: \emph{Wilson Hall 1-144}\\
	&  \\
	\hline
	\end{tabular*}
	
\vspace{2mm}
	


\hypertarget{course-description}{%
\section{Course Description}\label{course-description}}

Statistical methods of spatial data analysis, stationary and
nonstationary random fields, covariance structures, geostatistical
models and analysis, spatial point process models and analysis, spatial
lattice models and analysis. An emphasis will be placed on:

\begin{enumerate}
\def\labelenumi{\arabic{enumi}.}
\item
  Creating maps and other data visualization products with spatial data,
\item
  Identifying differences between the three common spatial data types:
  point process, geostatistical, and areal data,
\item
  Using statistical software and either Bayesian or classical
  statistical techniques to analyze spatial point process,
  geostatistical, and areal data structures, and
\item
  Implementing version control tools, such as git and github, on spatial
  data analyses.
\end{enumerate}

\hypertarget{learning-outcomes}{%
\section{Learning Outcomes:}\label{learning-outcomes}}

At the end of the course students will understand

\begin{enumerate}
\def\labelenumi{\arabic{enumi}.}
\item
  point process theory and applications including homogeneous and
  non-homogeneous Poisson point processes
\item
  geostatistics including semivariogram estimation and kriging
\item
  spatial autoregression including covariance estimation, spatial
  logistic and Poisson models, simultaneous autoregressive models,
  conditional autoregressive models.
\end{enumerate}

\hypertarget{prerequisites}{%
\section{Prerequisites}\label{prerequisites}}

\begin{itemize}
\tightlist
\item
  Required: STAT 412, STAT 512, and STAT 422
\item
  Preferred: STAT 506, extensive experience with R, and an understanding
  or interest in Bayesian statistics
\end{itemize}

\hypertarget{textbooks}{%
\section{Textbooks}\label{textbooks}}

\begin{itemize}
\tightlist
\item
  Hierarchical Modeling and Analysis for Spatial Data, Second Edition,
  by Bannerjee, Carlin, and Gelfand. While the second edition is
  preferred, the first edition will suffice.
\item
  Animal Movement: Statistical Models for Telemetry Data, by Hooten,
  Johnson, McClintock, and Morales. \emph{Optional}
\end{itemize}

\hypertarget{additional-resources}{%
\section{Additional Resources}\label{additional-resources}}

Analysis, data visualization, and version control procedures will be
implemented with:

\begin{itemize}
\tightlist
\item
  R / R Studio
\item
  Git / Github
\end{itemize}

\hypertarget{course-policies}{%
\section{Course Policies}\label{course-policies}}

\hypertarget{grading-policy} of your grade will be determined by homework
  assignments. Collaboration is encouraged on homework assignments, but
  everyone should complete their own assignments.
\item
  \textbf{25\%} of your grade will be determined by a midterm project.
\item
  \textbf{25\%} of your grade will be determined by a final project.
\end{itemize}

\hypertarget{collaboration}{%
\subsubsection{Collaboration}\label{collaboration}}

University policy states that, unless otherwise specified, students may
not collaborate on graded material. Any exceptions to this policy will
be stated explicitly for individual assignments. If you have any
questions about the limits of collaboration, you are expected to ask for
clarification.

In this class students are encouraged to collaborate on homework
assignments, but exams and projects should be completed without
collaboration.

\hypertarget{academic-misconduct}{%
\subsubsection{Academic Misconduct}\label{academic-misconduct}}

Section 420 of the Student Conduct Code describes academic misconduct as
including but not limited to plagiarism, cheating, multiple submissions,
or facilitating others' misconduct. Possible sanctions for academic
misconduct range from an oral reprimand to expulsion from the
university.

\hypertarget{disabilities-policy}{%
\subsubsection{Disabilities Policy}\label{disabilities-policy}}

Federal law mandates the provision of services at the university-level
to qualified students with disabilities. If you have a documented
disability for which you are or may be requesting an accommodation(s),
you are encouraged to contact the Office of Disability Services as soon
as possible.

\hypertarget{masks}{%
\subsubsection{Masks}\label{masks}}

WEARING MASKS IN CLASSROOMS IS REQUIRED Face coverings that cover the
mouth and nose are required in all indoor spaces and all enclosed or
partially enclosed outdoor spaces. MSU requires all students to wear
face masks or cloth face coverings in classrooms, laboratories and other
similar spaces where in-person instruction occurs. MSU requires the
wearing of masks in physical classrooms to help mitigate the
transmission of SARS-CoV-2, which causes COVID-19. The MSU community
views the adoption of these practices as a mark of good citizenship and
respectful care of fellow classmates, faculty, and staff.

The complete details about MSU's mask requirement can be found at
\url{https://www.montana.edu/health/coronavirus/index.html}.

These requirements from the Office of the Commissioner of Higher
Education are detailed in the MUS Healthy Fall 2020 Guidelines, Appendix
B.

For more information:
\url{https://www.montana.edu/health/coronavirus/prevention/index.html}

Compliance with the face-covering protocol is expected. If a you do not
comply with a classroom rule, you may be requested to leave class.
Section 460.00 of the MSU Code of Student Conduct covers ``disruptive
student behavior.''

\hypertarget{health-related-class-absences}{%
\subsubsection{Health-Related Class
Absences}\label{health-related-class-absences}}

Please evaluate your own health status regularly and refrain from
attending class and other on-campus events if you are ill.~MSU students
who miss class due to illness will be given opportunities to access
course materials online. You are encouraged to seek appropriate medical
attention for treatment of illness. In the event of contagious illness,
please do not come to class or to campus to turn in work. Instead notify
me by email about your absence as soon as practical, so that
accommodations can be made. Please note that documentation (a Doctor's
note) for medical excuses is not required. MSU University Health
Partners - as part their commitment to maintain patient confidentiality,
to encourage more appropriate use of healthcare resources, and to
support meaningful dialogue between instructors and students - does not
provide such documentation.

\hypertarget{course-communication}{%
\subsubsection{Course Communication}\label{course-communication}}

In the event that one or more students and/or the instructor are
required to quarantine or if the university moves courses online, the
course may need to continue in a virtual format. Communication about how
the course will proceed will be available through D2l.

\hypertarget{virtual-attendance}{%
\subsubsection{Virtual Attendance}\label{virtual-attendance}}

Due to the ongoing pandemic and issues stemming from this, a synchronous
virtual attendance option will be permitted for this course. The
Microsoft Teams platform will be used for this virtual option. When
attending virtually, if at all possible, please plan to have your video
camera turned on.

\hypertarget{approximate-course-outline}{%
\section{Approximate Course Outline}\label{approximate-course-outline}}

\begin{enumerate}
\def\labelenumi{\arabic{enumi}.}
\tightlist
\item
  Course Intro \& Preliminaries:
\end{enumerate}

\begin{itemize}
\tightlist
\item
  R
\item
  Git
\item
  Plotting spatial data
\item
  Linear Models
\item
  Stan / Bayesian Inference
\end{itemize}

\begin{enumerate}
\def\labelenumi{\arabic{enumi}.}
\setcounter{enumi}{1}
\tightlist
\item
  Gaussian Processes in 1D
\item
  Point Referenced Data
\item
  Areal Data
\item
  Point Process Data (and potentially animal movement models)
\end{enumerate}




\end{document}

\makeatletter
\def\@maketitle{%
  \newpage
%  \null
%  \vskip 2em%
%  \begin{center}%
  \let \footnote \thanks
    {\fontsize{18}{20}\selectfont\raggedright  \setlength{\parindent}{0pt} \@title \par}%
}
%\fi
\makeatother